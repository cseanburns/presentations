\documentclass[12pt]{article}
\usepackage{url}
\usepackage[margin=1.75in]{geometry}
\author{C.\ Sean Burns, University of Missouri}
\title{A Prolegomena to a Statistical Phenomenology or
    Phenomenometrics of the Couch; or An Investigation of the
    Experience of Sitting on the Couch; or What the Distance
    between Two People Sitting on a Couch Says about Their
    Relationship; and Other Thoughts}
\date{30 October 2012}
\begin{document}
\maketitle

\marginpar{slide: twitter info}

\section{What is Phenomenology?}

\marginpar{slide: section 1 page}

Phenomenology is the branch of philosophy, originally developed by
Edmund Husserl, that deals with the essence of objects in relation
to how we perceive those objects. How phenomenon appears to us has
a direct relationship to how we perceive it, and we can only know
the thing \textit{in and of itself} if we are able to bracket
those preconceived notions, put them away on a shelf somewhere,
close the door on the shelf, forget about it for a while, then
look, again, at the object in question with newness, with pure
consciousness.

\marginpar{slide: edmund husserl picture}

\section{A Brief History of the Couch and What it Symbolizes}

\marginpar{slide: section 2 page}

What is a couch? The couch dates back thousands of years to the
Arabians, who used it as a sort of throne. It was considered an
elite piece of furniture until the industrial revolution became
widespread and the lowly, leveling bourgeois were able to afford
one for their new living rooms in their new suburbs.

The couch has played a prominent role in 20th century history,
especially in relation to its ubiquitous counterpart, the
television, and later, the remote control, but also to potato
chips, hence the term, ``couch potato''.

The couch is a symbol of many things, oftentimes these things are
conflicting. The couch is a symbol of newly budding romance. It is
a symbol of the destruction of the libido. The couch is a symbol
of community. It is a symbol of isolation. The couch is a symbol
of relaxation. It is a symbol of sloth. Some couches are
comfortable. Some are not.

\section{Couch Sitting: A Phenomenometrics of Relationships}

\marginpar{slide: section 3 page}

How two people sit on a couch in quantitative relation to each
other \textit{unconceals} much truth about the relationship
between these two people. A standard couch yields approximately
6.5 feet of sitting space. The standard couch will hold three
standard people (Love-seats hold two standard people and are the
subject of another essay entitled, ``Love-seats: An Inquiry into
the Modern Art of Crossing Legs'', see Burns, \textit{Journal of
Ottoman Research and Development}, August 28, 2002, Volume III,
Issue 16).

\marginpar{slide: disclaimer page}

The average width of a modern American male is 18 inches or about
1.5 feet, not including arm space. The average width of a modern
American female is 14 inches, just two inches past a foot, not
including arm space.  Arm space for men adds another 7.3 inches
resulting in a total average width of 25.3 inches, and for women
arm space adds an extra 5.7 inches providing an average span of
19.7 inches.

There are generally three cushions on the standard couch. Given
that the standard couch is 6.5 feet wide, in sitting space, and
given that the standard couch holds three people, this provides
(6.5 / 3) feet, or 2.17 feet per person, or 26.0 inches. If the
standard American male possesses a span of 25.3 inches, this
standard American male will have a difference of 26.0 and 25.3
inches of room to spare. If the standard American male sits
centered on the cushion territory he occupies, this will provide
him with 0.35 inches on either side for sitting room adjustment.
More space will be allowed if the standard American male sits on
either the far left of the couch or far right of the couch,
allowing the standard American male to lean into the arm-rest and
provide additional space between him and the centered sitting
standard American male/female. For the standard American female,
the average room for adjustment for a centered sitting position is
the difference of 26.0 and 19.7, or 3.34 inches on either side.

To understand the relationship between two people sitting on a
couch together, one must analyze the difference in space,
preferably by feet, oftentimes by inches, between the two people
sitting on the couch together. (There is a separate analysis of
how three people sit on a standard three seater couch, but that is
the subject of another essay entitled, ``How Three People Sit on a
Couch Says about Them, or A Comparison between a Three People
Sitted Couch and Three Card Monty, a.k.a., Follow the Lady or also
Find the Lady,'' see Burns, (2004), \textit{Journal of Advanced
Furniture Design}, II(4)). In this analysis, we will take as
granted that each male and each female measures the standard
American width/span as defined above.

It is not debatable that if two people, of either sex (male/male,
female/male, male/female, female/female) sit on opposite sides
(respectively: +/- 27.14 inches apart, +/- 33 inches apart, -/+ 33
inches apart, -/+ 38.6 inches apart) of the couch, then these two
people are not in a romantic relationship, unless, that is, they
are undergoing a spat and if they are undergoing a spat, they are
alone, or only in the presence of ``comfortable'' company, that
is, company they are comfortable having a spat in front of, for
example, their children. If they are having a spat but they are in
the presence of formal company, they will sit much more closely
too each other (up to +/- 7.0 inches apart depending on cushion
centering and arm-rest leaning), not being comfortable revealing
their spat, at least directly, to this ``uncomfortable'' company.
The reason for this is because of our Puritan heritage, a heritage
that urges us to repress our feelings and feel ashamed of
outwardly displays of emotion and/or an inclination towards
secrecy and privacy.

Consequently, if two people sit on opposite sides of the couch,
several possibilities emerge: 

\begin{itemize}

\item They do not know each other and have just recently been
introduced;

\item They know each other, they are of either sex, and they are
comfortable with each other, but one of them wants to extend their
legs out into the empty space between them; 

\item The middle cushion is missing and so they have no choice but
to sit at either ends.

\end{itemize}

Furthermore, it is not debatable, studies have shown, that if two
people are sitting side by side, but with the standard space
between them, as defined above, then the relationship between
these two people are of one of the following natures: 

\begin{itemize}

\item They know each other and are friends of some intimacy,
whether physical or not; 

\item They have been married a long time and no longer share any
intimacy between each other but for the purpose of appearances and
convention, they sit side by side; 

\item One of the end cushions from the couch is missing (or
damaged or wet) and they have no choice but to sit side by side.

\end{itemize}

It is truly not necessary to explore the romantic positions. This
obviously implies that two people, of either sex, sit extremely
close together on the couch, ignoring any space between them,
touching each other, leg to leg, arm to arm, and so forth. Studies
have shown that this is oftentimes a situation that occurs between
two people sharing a new romance. This kind of sitting usually
fades after a year in the relationship.

\section{Future Research Areas}

\marginpar{slide: section 4 page}

Let me for a moment step away from \emph{phenomenometrics} and
discuss some methodological considerations.

\subsection{The Ottoman: Introducing Game Theoretical Dynamics}

\marginpar{slide: section 4.1 page}

An additional line of inquiry involves the application of game
theory to the study of couch sitting dynamics. This is normally
modeled in controlled, experimental situations by introducing an
Ottoman, with dimensions not greater than 2 by 2 feet, to a
scenario where two people are sharing a couch. The introduction of
an Ottoman to the scene drastically changes the dynamics of two
people sitting on a couch together.  One of three outcomes will
emerge: 

\begin{itemize}

\item One of the sitters will take an aggressive role and will
    preemptively take possession of the Ottoman for her or his own
    personal use; 

\item Each of the sitters will share the Ottoman by each placing a
    single foot on it or; 

\item Both, out of a sense decorum and respect, will ignore the
    Ottoman and leave it unused.

\end{itemize}

These are the noted standard events. There are other more rare
events that may occur in certain types of situations. These events
are of an explicit socio-biological nature and reveal the tendency
towards alpha-male behavior often observed in various gorilla
colonies. We often exclude these events as outliers, as they are,
fortunately, rare and brutal.

Despite the outliers, \emph{couch theorists} generally hold the
above outcomes constant as they alter other parts of the game
dynamic, such as the strategies and the payoffs.  As such, the
popular game models these theorists apply include the
\emph{footrest dilemma} and the \emph{coin in the cushion hunt}.

\subsection{The Couch vs.\ the Sofa: History and Future Trends}

\marginpar{slide: section 4.2 page}

For those of you who may be interested in \emph{couch science},
there are several lines of research to pursue.
\emph{Couchomatics}, for example, studies the intersection of
couch and technology use. Objects under investigation include the
use of television remote controls while sitting on a couch, alone
or with others, the use of game controllers in similar situations,
and so forth.

Additionally you could also pursue \emph{sofa science}, a sister
field. There was a point in time when \emph{couch} and \emph{sofa}
science were one and the same thing, but about thirty years ago a
bright young researcher found a statistically significant
difference between \emph{couch sitting behavior} and \emph{sofa
sitting behavior} among couples in long-term relationships.
Specifically, couches tend to have less cushion and a more upright
back while sofas tend to have more cushion and are made out of
softer material. Sofas, therefore, promoted individual relaxation
but at a cost of too much sloth, which damaged interpersonal
relations with the significant others. Unfortunately, this
difference resulted in different lines of inquiry, different
fields of research, and eventually, different academic
departments. Despite that, collaboration among researchers in
these two fields is increasing and recent findings suggest fresh
insights about \emph{sitting behavior} in general.

\section{Conclusion}

\marginpar{slide: section 5 page}

What does phenomenology say, in quantitative fashion, about the
couch and its role in American culture? We have seen that the
couch and how people sit on it says a manifold of things about the
relationship between those people, the types of people these
people are, whether they have character, where they are from. In
fact, the couch is so packed with preconceived notions, notions so
ingrained in our minds, that it is a most difficult activity to
bracket how we sit on the couch with each other and look at the
couch from a stance of pure, numerical consciousness.

However, it can be done. One of the ways bracketing our experience
of the couch can be accomplished is by challenging the static
notions we have inherited of the couch and also by disregarding
social norms such as the proper distance between two people no
matter what type of relationship they share with each other. By
simply sitting in close proximity with a complete stranger, the
sheer force of discomfort will so alter one's experience of couch
dynamics, that one will take a step closer to being able to
bracket these experiences and look at the couch anew. The key is
to keep present in the mind the role of the couch in such
engagement and always keep a can of pepper spray ready in case
things undesirably get out of hand.

\marginpar{slide: Questions and Answers Slide}

\section{Disclaimer}

The use of the U.S.\ Customary Units (feet, inches) to delineate
the measurements above is not the preferred measurement system.
The preferred measurement system is the International System of
Units, which uses the Metric System. Unfortunately, the preferred
system could not be used due to lack of resources to convert the
Americo-centric scientific data. Funds have been unavailable for
some time now. If you are interested in contributing to the
Movement for Metric Measurements, please visit
\url{http://www.mmmetric.com}.

\end{document}
